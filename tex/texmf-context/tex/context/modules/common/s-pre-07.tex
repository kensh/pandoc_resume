%D \module
%D   [      file=s-pre-07,
%D        version=1999.08.20,
%D          title=\CONTEXT\ Style File,
%D       subtitle=Presentation Environment 7,
%D         author=Hans Hagen,
%D           date=\currentdate,
%D      copyright={PRAGMA ADE \& \CONTEXT\ Development Team}]
%C
%C This module is part of the \CONTEXT\ macro||package and is
%C therefore copyrighted by \PRAGMA. See mreadme.pdf for
%C details.

%D This style was made for the \NTS\ presentation at
%D \EUROTEX\ 1999. It's a wink to programming in a webbed way.
%D This is just one way of implementing such a style. Today
%D we have more \METAPOST\ interfacing available, and
%D thereby moore tools and alternative ways to reach such a
%D goal. I must admit that the main macro looks fuzzy. On
%D the other hand, the presentation can look quite structured.
%D
%D \starttyping
%D \Topics{...}
%D
%D \StartIdeas
%D   \Topic{...}
%D   \StartIdea ... \StopIdea
%D   \StartIdea ... \StopIdea
%D \StopIdeas
%D \stoptyping

\startmode[asintended] \setupbodyfont[lbr] \stopmode

\setupbodyfont[14.4pt]

\usemodule
  [abr-02]

\setuppapersize
  [S6][S6]

\setuplayout
  [topspace=0cm,
   backspace=0cm,
   header=0pt,
   footer=0pt,
   width=middle,
   height=middle]

\setupinteractionscreen
  [option=max]

%D In order to prevent loops due to random placement, we
%D keep the random seed reasonable constant.

\setupsystem
  [random=big]

\setupcolors
  [state=start]

\definecolor[gray]     [s=.4]
\definecolor[lightgray][s=.9]

\definecolor[red]  [r=.4] \definecolor[cyan]   [g=.4,b=.4]
\definecolor[green][g=.4] \definecolor[magenta][r=.4,b=.4]
\definecolor[blue] [b=.4] \definecolor[yellow] [r=.4,g=.4]

\definecolor[PageColor][gray]
\definecolor[TextColor][lightgray]
\definecolor[LineColor][yellow]

\definecolor[linecolor 1][red]   \definecolor[linecolor 5][cyan]
\definecolor[linecolor 2][green] \definecolor[linecolor 6][magenta]
\definecolor[linecolor 3][blue]  \definecolor[linecolor 4][yellow]

\setupinteraction
  [state=start,
   display=new,
   color=LineColor,
   contrastcolor=LineColor]

\startuseMPgraphic{shape}
  path p ; color c, w ; numeric width, height ;
  c := \MPcolor{LineColor} ; w := \MPcolor{TextColor} ;
  width  := \overlaywidth ; height := \overlayheight ;
  pickup pencircle scaled .5cm ;
  p := unitcircle
    xscaled \MPw{\Idea} yscaled \MPh{\Idea}
    shifted \MPxy{\Idea}  ;
  for z = (0,.5height), (width,.5height), (.5width,0), (.5width,height),
          (0,0),        (width,height),   (0,height),  (width,0) :
    draw center p -- z withcolor c ;
  endfor ;
  fill p withcolor w ;
  draw p withcolor c ;
  p := unitcircle
    xscaled \MPw{\Page} yscaled \MPh{\Page}
    shifted \MPxy{\Page}  ;
  pickup pencircle scaled .25cm ;
  fill p withcolor w ;
  draw p withcolor c ;
  draw unitsquare xscaled width yscaled height withcolor c ;
\stopuseMPgraphic

\defineoverlay [shape]        [\useMPgraphic{shape}]
\defineoverlay [nextpage]     [\overlaybutton{nextpage}]
\defineoverlay [previouspage] [\overlaybutton{previouspage}]
\defineoverlay [content]      [\overlaybutton{content}]
\defineoverlay [forward]      [\overlaybutton{forward}]

\setupbackgrounds
  [page]
  [background={color,previouspage,shape},
   backgroundcolor=PageColor]

\def\StartIdea%
  {\xdef\Idea{idea:\realfolio}
   \xdef\Page{page:\realfolio}
   \startstandardmakeup
     \dontcomplain
     \vbox to \makeupheight \bgroup
       \getrandomdimen\scratchdimen{75pt}{600pt}\vskip 0pt plus \scratchdimen
       \hbox to \makeupwidth \bgroup
         \getrandomdimen\scratchdimen{75pt}{600pt}\hskip 0pt plus \scratchdimen
         \hpos{idea:\realfolio} \bgroup
           \framed
             [width=.6\hsize,height=fit,offset=2cm,align=middle,
              frame=off,strut=no,background=forward]
             \bgroup
             \setupwhitespace[big]}

\def\StopIdea%
            {\egroup
         \egroup
         \getrandomdimen\scratchdimen{75pt}{600pt}\hskip 0pt plus \scratchdimen
       \egroup
       \getrandomdimen\scratchdimen{75pt}{600pt}\vskip 0pt plus \scratchdimen
     \egroup
     \ifx\CurrentTopic\empty \else
       \vskip-\makeupheight
       \vbox to \makeupheight
         {\vfill
          \ifx\CurrentListTopic\empty\else
            \writetolist[Topic]{}{\CurrentListTopic}
          \fi
          \hbox to \makeupwidth
            {\hfill
             \hpos{page:\realfolio}
               {\framed
                  [offset=.5cm,frame=off,background=content]
                  {\bf\ignorespaces\CurrentTopic\unskip}}%
             \hskip.5cm}
          \vskip.5cm}
     \fi
   \stopstandardmakeup
   \let\CurrentListTopic\empty}

\definelist
  [Topic]

\setuplist
  [Topic]
  [alternative=f,
   expansion=command]

\let\CurrentTopic\empty
\let\CurrentListTopic\empty

\long\def\StartTopic#1\StopTopic
  {\long\def\CurrentTopic{#1}
   \let\CurrentListTopic\CurrentTopic}

\def\Topic#1%
  {\StartTopic#1\StopTopic}

\def\Topics#1%
  {\StartIdeas
     \def\CurrentTopic{#1}
     \StartIdea
       \pagereference[content]
       \placelist[Topic][criterium=all]
     \StopIdea
   \StopIdeas}

\newcounter\CurrentIdeas

\def\StartIdeas%
  {\ifnum\CurrentIdeas=6 \doglobal\newcounter\CurrentIdeas \fi
   \doglobal\increment\CurrentIdeas
   \definecolor[LineColor][linecolor \CurrentIdeas]}

\def\StopIdeas%
  {}

\doifnotmode{demo}{\endinput}

%D The (rather silly) demo section.

\starttext

\Topics{This is about \unknown}

\StartIdeas
    \Topic{Some topic}
    \StartIdea An idea \unknown \StopIdea
    \StartIdea \unknown\ and another \StopIdea
\StopIdeas

\stoptext


