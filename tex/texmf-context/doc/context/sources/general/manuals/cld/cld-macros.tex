% language=uk

\startcomponent cld-macros

\environment cld-environment

\startchapter[title=Macros]

\startsection[title={Introduction}]

You can skip this chapter if you're not interested in defining macros or are
quite content with defining them in \TEX. It's just an example of possible future
interface definitions and it's not the fastest mechanism around.

\stopsection

\startsection[title={Parameters}]

Right from the start \CONTEXT\ came with several user interfaces. As a
consequence you need to take this into account when you write code that is
supposed to work with interfaces other than the English one. The \TEX\ command:

\starttyping
\setupsomething[key=value]
\stoptyping

and the \LUA\ call:

\starttyping
context.setupsomething { key = value }
\stoptyping

are equivalent. However, all keys at the \TEX\ end eventually become English, but
the values are unchanged. This means that when you code in \LUA\ you should use
English keys and when dealing with assigned values later on, you need to
translate them of compare with translations (which is easier). This is why in the
\CONTEXT\ code you will see:

\starttyping
if somevalue == interfaces.variables.yes then
  ...
end
\stoptyping

instead of:

\starttyping
if somevalue == "yes" then
  ...
end
\stoptyping

\stopsection

\startsection[title={User interfacing}]

Unless this is somehow inhibited, users can write their own macros and this is
done in the \TEX\ language. Passing data to macros is possible and looks like
this:

\starttyping
\def\test#1#2{.. #1 .. #2 .. }      \test{a}{b}
\def\test[#1]#2{.. #1 .. #2 .. }    \test[a]{b}
\stoptyping

Here \type {#1} and \type {#2} represent an argument and there can be at most 9
of them. The \type{[]} are delimiters and you can delimit in many ways so the
following is also right:

\starttyping
\def\test(#1><#2){.. #1 .. #2 .. }  \test(a><b)
\stoptyping

Macro packages might provide helper macros that for instance take care of
optional arguments, so that we can use calls like:

\starttyping
\test[1,2,3][a=1,b=2,c=3]{whatever}
\stoptyping

and alike. If you are familiar with the \CONTEXT\ syntax you know that we use
this syntax all over the place.

If you want to write a macro that calls out to \LUA\ and handles things at that
end, you might want to avoid defining the macro itself and this is possible.

\startbuffer
\startluacode
function test(opt_1, opt_2, arg_1)
    context.startnarrower()
    context("options 1: %s",interfaces.tolist(opt_1))
    context.par()
    context("options 2: %s",interfaces.tolist(opt_2))
    context.par()
    context("argument 1: %s",arg_1)
    context.stopnarrower()
end

interfaces.definecommand {
    name = "test",
    arguments = {
        { "option", "list" },
        { "option", "hash" },
        { "content", "string" },
    },
    macro = test,
}
\stopluacode

test: \test[1][a=3]{whatever}
\stopbuffer

An example of a definition and usage at the \LUA\ end is:

\typebuffer

The call gives:

\startpacked
\getbuffer
\stoppacked

\startbuffer
\startluacode
local function startmore(opt_1)
    context.startnarrower()
    context("start more, options: %s",interfaces.tolist(opt_1))
    context.startnarrower()
end

local function stopmore(opt_1)
    context.stopnarrower()
    context("stop more, options: %s",interfaces.tolist(opt_1))
    context.stopnarrower()
end

interfaces.definecommand ( "more", {
    environment = true,
    arguments = {
        { "option", "list" },
    },
    starter = startmore,
    stopper = stopmore,
} )
\stopluacode

more: \startmore[1] one \startmore[2] two \stopmore one \stopmore
\stopbuffer

If you want to to define an environment (i.e.\ a \type {start}||\type {stop}
pair, it looks as follows:

\typebuffer

This gives:

\startpacked
\getbuffer
\stoppacked

The arguments are know in both \type {startmore} and \type {stopmore} and nesting
is handled automatically.

\stopsection

\startsection[title=Looking inside]

If needed you can access the body of a macro. Take for instance:

\startbuffer
\def\TestA{A}
\def\TestB{\def\TestC{c}}
\def\TestC{C}
\stopbuffer

\typebuffer \getbuffer

The following example demonsttaies how we can look inside these macros. You need
to be aware of the fact that the whole blob of \LUA\ codes is finished before we
return to \TEX, so when we pipe the meaning of \type {TestB} back to \TEX\ it
only gets expanded afterwards. We can use a function to get back to \LUA. It's
only then that the meaning of \type {testC} is changed by the (piped) expansion
of \type {TestB}.

\startbuffer
\startluacode
context(tokens.getters.macro("TestA"))
context(tokens.getters.macro("TestB"))
context(tokens.getters.macro("TestC"))
tokens.setters.macro("TestA","a")
context(tokens.getters.macro("TestA"))
context(function()
    context(tokens.getters.macro("TestA"))
    context(tokens.getters.macro("TestB"))
    context(tokens.getters.macro("TestC"))
end)
\stopluacode
\stopbuffer

\typebuffer \getbuffer

Here is another example:

\startbuffer
\startluacode
if tokens.getters.macro("fontstyle") == "rm" then
    context("serif")
else
    context("unknown")
end
\stopluacode
\stopbuffer

\typebuffer

Of course this assumes that you have some knowledge of the \CONTEXT\ internals.

\getbuffer

\stopsection

\stopchapter

\stopcomponent
