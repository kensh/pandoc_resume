% language=uk

\startcomponent cld-afewdetails

\environment cld-environment

\startchapter[title=A few Details]

\startsection[title=Variables]

\index{user interface}

Normally it makes most sense to use the English version of \CONTEXT. The
advantage is that you can use English keywords, as in:

\starttyping
context.framed( {
    frame = "on",
  },
  "some text"
)
\stoptyping

If you use the Dutch interface it looks like this:

\starttyping
context.omlijnd( {
    kader = "aan",
  },
  "wat tekst"
)
\stoptyping

A rather neutral way is:

\starttyping
context.framed( {
    frame = interfaces.variables.on,
  },
  "some text"
)
\stoptyping

But as said, normally you will use the English user interface so you can forget
about these matters. However, in the \CONTEXT\ core code you will often see the
variables being used this way because there we need to support all user
interfaces.

\stopsection

\startsection[title=Modes]

\index{modes}
\index{systemmodes}
\index{constants}

Context carries a concept of modes. You can use modes to create conditional
sections in your style (and|/|or content). You can control modes in your styles
or you can set them at the command line or in job control files. When a mode test
has to be done at processing time, then you need constructs like the following:

\starttyping
context.doifmodeelse( "screen",
  function()
      ... -- mode == screen
  end,
  function()
      ... -- mode ~= screen
  end
)
\stoptyping

However, often a mode does not change during a run, and then we can use the
following method:

\starttyping
if tex.modes["screen"] then
  ...
else
  ...
end
\stoptyping

Watch how the \type {modes} table lives in the \type {tex} namespace. We also
have \type {systemmodes}. At the \TEX\ end these are mode names preceded by a
\type {*}, so the following code is similar:

\starttyping
if tex.modes["*mymode"] then
  -- this is the same
elseif tex.systemmodes["mymode"] then
  -- test as this
else
  -- but not this
end
\stoptyping

Inside \CONTEXT\ we also have so called constants, and again these can be
consulted at the \LUA\ end:

\starttyping
if tex.constants["someconstant'] then
  ...
else
  ...
end
\stoptyping

But you will hardly need these and, as they are often not public, their
meaning can change, unless of course they {\em are} documented as public.

\stopsection

\startsection[title={Token lists}]

\index{tokens}

There is normally no need to mess around with nodes and tokens at the \LUA\ end
yourself. However, if you do, then you might want to flush them as well. Say that
at the \TEX\ end we have said:

\startbuffer
\toks0 = {Don't get \inframed{framed}!}
\stopbuffer

\typebuffer \getbuffer

Then at the \LUA\ end you can say:

\startbuffer
context(tex.toks[0])
\stopbuffer

\typebuffer

and get: \ctxluabuffer\ In fact, token registers are exposed as strings so here,
register zero has type \type {string} and is treated as such.

\startbuffer
context("< %s >",tex.toks[0])
\stopbuffer

\typebuffer

This gives: \ctxluabuffer. But beware, if you go the reverse way, you don't get
what you might expect:

\startbuffer
tex.toks[0] = [[\framed{oeps}]]
\stopbuffer

\typebuffer \ctxluabuffer

If we now say \type{\the\toks0} we will get {\tttf \the\toks0} as
all tokens are considered to be letters.

\stopsection

\startsection[title={Node lists}]

\index{nodes}

If you're not deep into \TEX\ you will never feel the need to manipulate node
lists yourself, but you might want to flush boxes. As an example we put something
in box zero (one of the scratch boxes).

\startbuffer
\setbox0 = \hbox{Don't get \inframed{framed}!}
\stopbuffer

\typebuffer \getbuffer

At the \TEX\ end you can flush this box (\type {\box0}) or take a copy
(\type{\copy0}). At the \LUA\ end you would do:

\starttyping
context.copy()
context.direct(0)
\stoptyping

or:

\starttyping
context.copy(false,0)
\stoptyping

but this works as well:

\startbuffer
context(node.copy_list(tex.box[0]))
\stopbuffer

\typebuffer

So we get: \ctxluabuffer\ If you do:

\starttyping
context(tex.box[0])
\stoptyping

you also need to make sure that the box is freed but let's not go into those
details now.

Here is an example if messing around with node lists that get seen before a
paragraph gets broken into lines, i.e.\ when hyphenation, font manipulation etc
take place. First we define some colors:

\startbuffer
\definecolor[mynesting:0][r=.6]
\definecolor[mynesting:1][g=.6]
\definecolor[mynesting:2][r=.6,g=.6]
\stopbuffer

\typebuffer \getbuffer

Next we define a function that colors nodes in such a way that we can see the
different processing stages.

\startbuffer
\startluacode
local enabled  = false
local count    = 0
local setcolor = nodes.tracers.colors.set

function userdata.processmystuff(head)
    if enabled then
        local color = "mynesting:" .. (count % 3)
     -- for n in node.traverse(head) do
        for n in node.traverse_id(nodes.nodecodes.glyph,head) do
            setcolor(n,color)
        end
        count = count + 1
        return head, true
    end
    return head, false
end

function userdata.enablemystuff()
    enabled = true
end

function userdata.disablemystuff()
    enabled = false
end
\stopluacode
\stopbuffer

\typebuffer \getbuffer

We hook this function into the normalizers category of the processor callbacks:

\startbuffer
\startluacode
nodes.tasks.appendaction("processors", "normalizers", "userdata.processmystuff")
\stopluacode
\stopbuffer

\typebuffer \getbuffer

We now can enable this mechanism and show an example:

\startbuffer
\startbuffer
Node lists are processed \hbox {nested from \hbox{inside} out} which is not
what you might expect. But, \hbox{coloring} does not \hbox {happen} really
nested here, more \hbox {in} \hbox {the} \hbox {order} \hbox {of} \hbox
{processing}.
\stopbuffer

\ctxlua{userdata.enablemystuff()}
\par \getbuffer \par
\ctxlua{userdata.disablemystuff()}
\stopbuffer

\typebuffer

The \type {\par} is needed because otherwise the processing is already disabled
before the paragraph gets seen by \TEX.

\blank \getbuffer \blank

\startbuffer
\startluacode
nodes.tasks.disableaction("processors", "userdata.processmystuff")
\stopluacode
\stopbuffer

\typebuffer

Instead of using an boolean to control the state, we can also do this:

\starttyping
\startluacode
local count    = 0
local setcolor = nodes.tracers.colors.set

function userdata.processmystuff(head)
    count = count + 1
    local color = "mynesting:" .. (count % 3)
    for n in node.traverse_id(nodes.nodecodes.glyph,head) do
        setcolor(n,color)
    end
    return head, true
end

nodes.tasks.appendaction("processors", "after", "userdata.processmystuff")
\stopluacode
\stoptyping

\startbuffer
\startluacode
nodes.tasks.disableaction("processors", "userdata.processmystuff")
\stopluacode
\stopbuffer

Disabling now happens with:

\typebuffer \getbuffer

As you might want to control these things in more details, a simple helper
mechanism was made: markers. The following example code shows the way:

\startbuffer
\definemarker[mymarker]
\stopbuffer

\typebuffer \getbuffer

Again we define some colors:

\startbuffer
\definecolor[mymarker:1][r=.6]
\definecolor[mymarker:2][g=.6]
\definecolor[mymarker:3][r=.6,g=.6]
\stopbuffer

\typebuffer \getbuffer

The \LUA\ code like similar to the code presented before:

\startbuffer
\startluacode
local setcolor    = nodes.tracers.colors.setlist
local getmarker   = nodes.markers.get
local hlist_code  = nodes.codes.hlist
local traverse_id = node.traverse_id

function userdata.processmystuff(head)
    for n in traverse_id(hlist_code,head) do
        local m = getmarker(n,"mymarker")
        if m then
            setcolor(n.list,"mymarker:" .. m)
        end
    end
    return head, true
end

nodes.tasks.appendaction("processors", "after", "userdata.processmystuff")
nodes.tasks.disableaction("processors", "userdata.processmystuff")
\stopluacode
\stopbuffer

\typebuffer \getbuffer

This time we disabled the processor (if only because in this document we don't
want the overhead.

\startbuffer
\startluacode
nodes.tasks.enableaction("processors", "userdata.processmystuff")
\stopluacode

Node lists are processed \hbox \boxmarker{mymarker}{1} {nested from \hbox{inside}
out} which is not what you might expect. But, \hbox {coloring} does not \hbox
{happen} really nested here, more \hbox {in} \hbox \boxmarker{mymarker}{2} {the}
\hbox {order} \hbox {of} \hbox \boxmarker{mymarker}{3} {processing}.

\startluacode
nodes.tasks.disableaction("processors", "userdata.processmystuff")
\stopluacode
\stopbuffer

\typebuffer

The result looks familiar:

\getbuffer

% We don't want the burden of this demo to cary on:

% {\em If there's enough interest I will expand this section with some basic
% information on what nodes are.}

\stopsection

\stopchapter

\stopcomponent
