\startcomponent ma-cb-en-margintexts

\enablemode[**en-us]

\project ma-cb

\startchapter[title=Margin texts]

\index{margin text}

\Command{\tex{margintext}}
\Command{\tex{inmargin}}
\Command{\tex{inleft}}
\Command{\tex{inright}}
\Command{\tex{margintitle}}

It is very easy to put text in the margin. You just use \type{\inmargin}.

\shortsetup{inmargin}

You may remember one of the earlier examples:

\startbuffer[marginpicture]
\inmargin
  {\externalfigure
     [ma-cb-23]
     [width=.6\marginwidth]}
\stopbuffer

\typebuffer[marginpicture]

This would result in a figure in the \pagereference [marginpicture]
margin. You can imagine that it looks quite nice in some
documents. But be careful. The margin is rather small so the figure could become
very marginal.

A few other examples are shown in the text below.\getbuffer
[marginpicture]

\startbuffer
The Ridderstraat (Street of knights) \inmargin{Street of\\Knights}
is an obvious name. In the 14th and 15th centuries, nobility and
prominent citizens lived in this street. Some of their big houses
were later turned into poorhouses \inright{poorhouse}and old
peoples homes.

Up until \inleft[low]{\tfc 1940}1940 there was a synagog in the
Ridderstraat. Some 40 Jews gathered there to celebrate their
sabbath. During the war all Jews were deported to Westerbork and
then to the extermination camps in Germany and Poland. None of
the Jewish families returned. The synagog was knocked down in
1958.
\stopbuffer

\typebuffer

The commands \type{\inmargin}, \type{\inleft} and \type{\inright} all have the
same function. In a two sided document \type{\inmargin} puts the margin text in
the correct margin. The \type{\\} is used for line breaking. The example above
would look like this:

\getbuffer

You can set up the margin text with:

\shortsetup{setupinmargin}

Other commands that you can use for forcing text into the margin
are listed in \in{table}[tab:margincommands].

\placetable
  [here]
  [tab:margincommands]
  {Overview of margin commands.}
  {\starttable[|l|l|]
  \HL
  \NC \bf Command           \NC \bf Meaning \NC\SR
  \HL
  \NC \type{\ininner}       \NC text in inner margin \NC\FR
  \NC \type{\inouter}       \NC text in outer margin \NC\MR
  \NC \type{\inright}       \NC text in right margin \NC\MR
  \NC \type{\inleft}        \NC text in left margin \NC\MR
  \NC \type{\inmargin}      \NC text in the margin \NC\MR
  \NC \type{\inothermargin} \NC text in other margin \NC\MR
  \NC \type{\margintext}    \NC text in the margin \NC\LR
  \HL
  \stoptable}

If you want to place more extensive text blocks in the margin there is the
command:

\shortsetup{marginblock}

and the accompanying command:

\shortsetup{setupmarginblocks}

\stopchapter

\stopcomponent
