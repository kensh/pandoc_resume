\startcomponent ma-cb-en-presentations

\enablemode[**en-us]

\project ma-cb

\startchapter[title=Presentations,reference=presentations]

\index{presentation}

You can use \CONTEXT\ for making your own presentations. A \CONTEXT\ presentation
is an interactive PDF document with a screen layout. Often presentations are good
examples of the cooperation between \CONTEXT\ and \METAPOST.

\CONTEXT\ comes with a number ready-to-use presentations. A presentation is a
module with the prefix \type{s-} and that you can load with the \type{\usemodule}
command.

If you want to use an already existing presentation the best way to proceed is:

\startitemize[packed]
\item goto \type{../your-contextdir/tex/texmf-context/tex/context/base} in your text editor
\item open a presentation: for example \type{s-pre-05.tex}
\item goto the end of the file and study the commands between
      the \type{\start...\stoptext} pair
\item copy the commands into your own presentation file
\item invoke the presentation with \type{\usemodule[s][pre-05]} in de setup
      area of your presentation file
\item process the file to view the result
\item edit the content of your presentation
\stopitemize

A stepwise setup of a presentation is given at the
\goto{\CONTEXTWIKI}[ url (http://wiki.contextgarden.net/Presentations) ].

\stopchapter

\stopcomponent
