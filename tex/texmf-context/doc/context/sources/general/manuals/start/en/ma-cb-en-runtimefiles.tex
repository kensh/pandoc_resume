\startcomponent ma-cb-en-runtimefiles

\enablemode[**en-us]

\project ma-cb

\startchapter[title=Auxilliary files,reference=runtimefiles]

\index[tuc]{{\tt tuc}--file}
\index{auxilliary files}

\CONTEXT\ will produce a number of auxilliary files during processing. If your
input file is called \type{myfile.tex} the following files may appear on your
working directory.

\index[tuc]{{\tt tuc}--file}
\index{auxilliary files}

\starttabulate[|l|l|l|]
\HL
\NC \darkgray \bf \CONTEXT\ MkII \NC \bf \CONTEXT\ MkIV \NC \bf Meaning                      \NC\NR
\HL
\NC \darkgray \tt myfile.tex     \NC \tt myfile.tex     \NC your text file                   \NC\NR
\HL
\NC \darkgray \tt myfile.log     \NC \tt myfile.log     \NC log information                  \NC\NR
\NC \darkgray \tt myfile.tuo     \NC \tt myfile.tuc     \NC output information               \NC\NR
\NC \darkgray \tt myfile.tui     \NC                    \NC \darkgray input information      \NC\NR
\NC \darkgray \tt myfile.tmp     \NC                    \NC \darkgray temporary information  \NC\NR
\NC \darkgray \tt mpgraph.mp     \NC                    \NC \darkgray \METAPOST\ information \NC\NR
\HL
\NC \darkgray \tt myfile.pdf     \NC \tt myfile.pdf     \NC result file                      \NC\NR
\HL
\stoptabulate

The \type{myfile.tuc} file contains information about registers, lists and
references which will be used when necessary. The \type{myfile.log}
can be viewed in case there are problems during processing.

\stopchapter

\stopcomponent
