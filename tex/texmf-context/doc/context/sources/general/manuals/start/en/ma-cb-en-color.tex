\startcomponent ma-cb-en-color

\project ma-cb

\startchapter[title=Color]

\index{color}

\Command{\tex{setupcolors}}
\Command{\tex{color}}
\Command{\tex{definecolor}}

Text, frames or backgrounds can be set in color with:

\shortsetup{color}

Default the basic colors are available. Basic colors are for example
red, white and blue. A color like orange can be defined with:

\shortsetup{definecolor}

You can define orange like this:

\startbuffer[a]
\definecolor [darkorange]   [c=0.0,m=0.60,y=1.00,k=0.0]
\definecolor [middleorange] [.5(darkorange)]
\stopbuffer

\typebuffer[a]

\getbuffer[a]

It is of good practice to check (combinations of) colors on a larger
surface:

\startbuffer
\blackrule[width=\hsize,height=1cm,color=red,after=]
\blackrule[width=\hsize,height=1cm,color=white,after=]
\blackrule[width=\hsize,height=1cm,color=blue,after=]
\blackrule[width=\hsize,height=1cm,color=darkorange]
\stopbuffer

\typebuffer

so you can see if they fit together:

\blank

\getbuffer

A color can be invoked in a number of ways:

\startbuffer
\startcolor[red]
On {\darkorange Kingsday} {\blue Hasselt} turns into a
\color[darkorange]{colorfull} city.
\stopcolor
\stopbuffer

\typebuffer

\getbuffer

More information on the use of color models, transparency and palets can be found
on the \goto {\CONTEXTWIKI} [ url (http://wiki.contextgarden.net/Color) ]
and in the
\goto {\em Color Separation} [ url (manual:color) ] manual.

\stopchapter

\stopcomponent
