\startcomponent ma-cb-en-synonyms

\enablemode[**en-us]

\project ma-cb

\startchapter[reference=synonyms,title=Synonyms]

\index{synonyms}

\Command{\tex{definesynonyms}}
\Command{\tex{setupsynonyms}}
\Command{\tex{abbreviation}}
\Command{\tex{infull}}
\Command{\tex{loadabbreviations}}
\Command{\tex{placelistofabbreviations}}
\Command{\tex{completelistofabbreviations}}

In many documents people want to use specific words consistently throughout the
document. To enforce consistency the command below is available.

\shortsetup{definesynonyms}

The first bracket pair contains the singular form of the synonym, and the second
contains the plural form. The third bracket pair contains a command.

For example the command \type{\abbreviation} is defined by:

\starttyping
\definesynonyms[abbreviation][abbreviations][\infull]
\setupsynonyms[style=cap]
\stoptyping

Now the command \type{\abbreviation} is available and can be used to state your
abbreviations:

\starttyping
\abbreviation{ANWB}{Dutch Automobile Association}
\abbreviation{VVV}{Bureau of Tourist Information}
\abbreviation{NS}{Dutch Railways}
\stoptyping

\abbreviation{VVV}{Bureau of Tourist Information}

If you would type:

\startbuffer
The Dutch \VVV\ (\infull{VVV}) can provide you with the tourist
information on Hasselt.
\stopbuffer

\typebuffer

You would obtain something like this:

\getbuffer

The list of synonyms or abbreviations is best defined in the set up area of your
input file for maintenance purposes. You can also store this kind of information
in an external file, and load the file (e.g. \type{abbrev.tex}) with:

\starttyping
\input abbrev.tex
\stoptyping

If you want to put a list of the abbreviations used in your document you can
type:

\starttyping
\placelistofabbreviations
\stoptyping

or

\starttyping
\completelistofabbreviations
\stoptyping

A complete and sorted list with used abbreviations and their meaning is produced.

The typesetting of synonynms can be influenced with:

\starttyping
\setupsynonyms
\stoptyping

\stopchapter

\stopcomponent
