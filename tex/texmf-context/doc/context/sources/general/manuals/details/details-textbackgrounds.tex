% language=uk

\environment details-environment

\startcomponent details-textbackgrounds

\start \setuphead [chapter] [after=] \startchapter[title={Backgrounds behind text}]

\startbuffer[setup-a]
\definetextbackground
  [intro]
  [backgroundcolor=infogray,
   backgroundoffset=.25cm,
   frame=off,
   location=paragraph,
   color=red]
\stopbuffer

\startbuffer[setup-b]
\definetextbackground
  [subintro]
  [backgroundcolor=textgray,
   backgroundoffset=0pt,
   frame=off,
   location=text,
   color=blue]
\stopbuffer

\startbuffer[demo-a]
\starttextbackground[intro]
A rather common way to draw attention to a passage, is to add a
background. In this chapter we will therefore discuss how to enhance your
document with \starttextbackground [subintro] those colorful areas that either
or not follow the shape of your paragraph. \stoptextbackground\ Be
warned: this chapter has so many backgrounds that you might start to
dislike them.
\stoptextbackground
\stopbuffer

\getbuffer[setup-a,setup-b,demo-a]

\blank

In the previous paragraph we demonstrated two important features of the
background handler: you can nest backgrounds and backgrounds can be tight or
wide. Features like this will often be used in combination with others, like
special section headers. The raw coding of the previous paragraph is therefore
not representative.

\typebuffer[demo-a]

The outer background commands is defined as follows:

\typebuffer[setup-a]

Here, the \type {paragraph} option ensures that the background covers the width
of the body text. The inner background is defined in a similar way, but this time
we choose \type {text} location.

\typebuffer[setup-b]

In this document we use protruding characters (hanging punctuation) so we've
chosen a rather large offset, one that also matches the rest of the page design.

Those who are familiar with the way \TEX\ works will probably see what problems
can occur with backgrounds like this. What happens for instance when we cross
page boundaries, and how will more complicated paragraph shapes be handled?

The current implementation tries to handle page breaks and paragraph shapes as
good as possible. This works well in normal one||column mode as well as in
columns.

\startbuffer[setup-c]
\definetextbackground [A] [backgroundcolor=infogray]
\definetextbackground [B] [backgroundcolor=textgray]

\setuptextbackground
  [backgroundoffset=0pt,
   offset=0pt,
   frame=off,
   location=text]
\stopbuffer

\getbuffer[setup-c]

\startbuffer[demo-b]
\placefigure[left]{}{\externalfigure[detcow][width=2cm]}

\starttextbackground [A]
  In this example, the paragraph shape is determined by the graphic placed
  left of the text.
    \starttextbackground [B]
      This feature is implemented using the \type {\hangindent} and \type
      {\hangafter} primitives, which means that we need to keep track of
      their state. In addition, we need to handle the indentation directives
      \type {\leftskip}, \type {\rightskip} and \type {\parindent}.
    \stoptextbackground\
  Because backgrounds end up in a different background overlay, nesting
  them is no problem, and it is even possible to move them to the front
  and back, as we will demonstrate later on. While the mechanism discussed
  here will always be improved when we find border cases, the fundaments
  it is built upon are quite stable.
\stoptextbackground
\stopbuffer

{\setupalign[nothanging]\getbuffer[demo-b]\par}

\typebuffer[demo-b]

The backgrounds were defined as:

\typebuffer[setup-c]

\startbuffer[setup-d]
\setuptextbackground [B] [backgroundcolor=darkgray,level=+2]
\stopbuffer

{\setupalign[nothanging]\getbuffer[setup-d,demo-b]\par}

This time we moved the inner background a few levels up. By default they reside
at \type {level=-1}. This way, by using a non transparent color, we can hide
information.

\typebuffer[setup-d]

Unless you mess around too much with boxes, backgrounds work as expected in most
situations. According to the Merriam||Webster on the authors laptop:

\startbuffer
\starttabulate[|l|p|l|]
\NC background \NC \starttextbackground [A] the part of a
    painting representing what lies behind objects is the
    \starttextbackground [B] foreground \stoptextbackground
    \stoptextbackground \NC one \NC \NR
\TB [halfline]
\NC foreground \NC \starttextbackground [A] the part of a
    scene or representation that is nearest to and in front
    of the \starttextbackground [B] spectator
    \stoptextbackground \stoptextbackground \NC two \NC \NR
\TB [halfline]
\NC spectator \NC  \starttextbackground [A] one who looks
    on or watches \stoptextbackground \NC three \NC \NR
\stoptabulate
\stopbuffer

\getbuffer

This is coded similar to normal running text. A table like this is in a way still
part of the text flow. As floating body (see \in {table} [tab:back]) it can
virtually end up everywhere. We add a frame to make clear where the boundaries are.

\start

    \setupfloat
      [table]
      [frame=on,framecolor=red,rulethickness=1pt]

    \placetable
      [here] [tab:back]
      {} {\hsize.75\textwidth\getbuffer}

    \definefloat
        [mytable]
        [table]

    \setupfloat
        [mytable]
        [leftmargindistance=-\innermargintotal]

    \placemytable
        [left,high,low]
        [tab:back-m]
        {}
        {\hsize.5\textwidth\getbuffer}

    Keeping track of the state of a paragraph in a table in combination with
    background is not entirely trivial. The current implementation evolved from
    less clever ones and, unless you start doing complicated box manipulations
    with the float content, works quite well. One reason why we made backgrounds
    work in tables (and especially floating tables) is that is was needed for
    typesetting books for primary and secundary education. In there, we want to
    be able to hide the answers that students are supposed to fill in.

    \flushsidefloats

\stop

In \in {figure} [fig:columns:1] you can see an advanced example of backgrounds
running over columns. If you look carefully, you will notice that the background
depends on the kind of background at hand:

\startitemize[n,packed]
\item the text starts and flows on
\item the text flows on (or stands alone)
\item the text flows on and ends
\stopitemize

This information is available when you want to draw your own backgrounds. Here
the graphic was defined as follows:

\startplacefigure [reference=fig:columns:1]
    \startcombination[4*1]
          {\externalfigure[back-4.pdf][page=1,width=\distributedhsize\textwidth\emwidth4]}{Page 1}
          {\externalfigure[back-4.pdf][page=2,width=\distributedhsize\textwidth\emwidth4]}{Page 2}
          {\externalfigure[back-4.pdf][page=3,width=\distributedhsize\textwidth\emwidth4]}{Page 3}
          {\externalfigure[back-4.pdf][page=4,width=\distributedhsize\textwidth\emwidth4]}{Page 4}
    \stopcombination
\stopplacefigure

\starttyping
\startuseMPgraphic{mpos:par:color}
  for i=1 upto nofmultipars :
    fill multipars[i] withcolor
      if     multikind[i]="single" : "darkgray" ;
      elseif multikind[i]="first"  : "red" ;
      elseif multikind[i]="middle" : "green" ;
      elseif multikind[i]="last"   : "blue" ;
      else                         : "black" ;
      fi ;
  endfor ;
\stopuseMPgraphic
\stoptyping

This graphic is hooked into the background setup by setting the \type {mp}
variable.

\starttyping
\definetextbackground
  [shade]
  [location=paragraph,
   mp=mpos:par:color,
   before=\blank,
   after=\blank]
\stoptyping

A variant is the following. This time we use a shade:

\starttyping
\startuseMPgraphic{mpos:par:columnset:shade}
  numeric h ;
  for i=1 upto nofmultipars :
    h := bbheight(p) ;
    if multikind[i] = "single" :
      fill multipars[i] topenlarged -.5h
        withshademethod "linear"
        withshadedirection shadedup
        withcolor boxfillcolor shadedinto .8white ;
      fill multipars[i] bottomenlarged -.5h
        withshademethod "linear"
        withshadedirection shadedup
        withcolor .8white shadedinto boxfillcolor ;
    elseif multikind[i] = "first" :
      fill multipars[i]
        withshademethod "linear"
        withshadedirection shadedup
        withcolor boxfillcolor shadedinto .8white ;
    elseif multikind[i] = "middle" :
      fill multipars[i] topenlarged -.5h
        withshademethod "linear"
        withshadedirection shadedup
        withcolor boxfillcolor shadedinto .8white ;
      fill multipars[i] bottomenlarged -.5h
        withshademethod "linear"
        withshadedirection shadedup
        withcolor .8white shadedinto boxfillcolor ;
    elseif multikind[i] = "last" :
      fill multipars[i]
        withshademethod "linear"
        withshadedirection shadedup
        withcolor .8white shadedinto boxfillcolor ;
    fi ;
  endfor ;
\stopuseMPgraphic
\stoptyping

When we hook it into the background we get \in {figure} [fig:columns:2] as result:

\starttyping
\definetextbackground
  [shade]
  [location=paragraph,
   backgroundcolor=shadecolor,
   mp=mpos:par:columnset:shade,
   before=\blank,
   after=\blank]
\stoptyping

\startplacefigure [reference=fig:columns:2]
    \startcombination[4*1]
          {\externalfigure[back-5.pdf][page=1,width=\distributedhsize\textwidth\emwidth4]}{Page 1}
          {\externalfigure[back-5.pdf][page=2,width=\distributedhsize\textwidth\emwidth4]}{Page 2}
          {\externalfigure[back-5.pdf][page=3,width=\distributedhsize\textwidth\emwidth4]}{Page 3}
          {\externalfigure[back-5.pdf][page=4,width=\distributedhsize\textwidth\emwidth4]}{Page 4}
    \stopcombination
\stopplacefigure

The complexity of the backgrounds mechanism is partly due to the fact that we
want to use arbitrary \METAPOST\ code to render the background. For instance, we
want to have a proper shape so that not only the filled shape but also the drawn
shape comes out right. You can compare this to a glyph in a font: when rendered
filled the outline can be anything as it will not be drawn but when we use the
outline we can run into overlaps and such. Where glyphs can use the odd|-|even
filling methods, background can only use that for simple cases.

When a background is rectangular it's all quite easy but as soon as some holes
occur we need to do more work. Holes can be the result of a image placed next to
the running text, or an image flushed at a page break or in the middle of a
background. Paragraph shapes are another example. Backgrounds can cross page
boundaries too. Yet another property is nesting and in such cases the shape is
a bit more complex as we cross lines partially.

In \MKII\ the background mechanism already was quite useable but it had some
limitations. Calculating the background was mostly delegated to \METAPOST\ which
is reasonable. In \MKIV\ some work is delegated to \LUA\ instead but that doesn't
mean that the code is cleaner or easier to understand. So, to summarize, there
are several cases that we need to take into account, like:

\startitemize
    \startitem
        A background can run behind a paragraph in which case the start is
        leftmost and end rightmost. In this case inserts (like floats) have to be
        dealt with after the shape has been calculated.
    \stopitem
    \startitem
        A background can be in|-|line (the \type {text} location variant) in
        which case we need to follow the paragraph shape, if set. In that case we
        have a mix of calculating the background shape and afterwards
        compensating for inserts.
    \stopitem
    \startitem
        A third case is tabulation and tables where we have dedicated regions to
        deal with. When these float we need to make sure that the backgrounds are
        adapted to the where they end up.
    \stopitem
    \startitem
        Yet another case is in columns, where we hape multiple regions to deal
        with.
    \stopitem
    \startitem
        As mentioned, floats need special treatment and they can be part of the
        page flow but also end up left or right of the text (either or not
        shifted) but also in the margins, edges, back- or cutspace. Their
        placement influences the way backgrounds are calculated so additional
        information needs to travel with them.
    \stopitem

\stopitemize

We distinguish between a paragraph background, which runs between the left and right skip
areas and a text background which follows a shape. In \in {figure} [fig:columns:3] we see a
test case with several such shapes.

\startplacefigure [reference=fig:columns:3]
    \startcombination[4*3]
          {\externalfigure[back-2.pdf][page=1, width=\distributedhsize\textwidth\emwidth4]}{Page  1}
          {\externalfigure[back-2.pdf][page=2, width=\distributedhsize\textwidth\emwidth4]}{Page  2}
          {\externalfigure[back-2.pdf][page=3, width=\distributedhsize\textwidth\emwidth4]}{Page  3}
          {\externalfigure[back-2.pdf][page=4, width=\distributedhsize\textwidth\emwidth4]}{Page  4}
          {\externalfigure[back-2.pdf][page=5, width=\distributedhsize\textwidth\emwidth4]}{Page  5}
          {\externalfigure[back-2.pdf][page=6, width=\distributedhsize\textwidth\emwidth4]}{Page  6}
          {\externalfigure[back-2.pdf][page=7, width=\distributedhsize\textwidth\emwidth4]}{Page  7}
          {\externalfigure[back-2.pdf][page=8, width=\distributedhsize\textwidth\emwidth4]}{Page  8}
          {\externalfigure[back-2.pdf][page=9, width=\distributedhsize\textwidth\emwidth4]}{Page  9}
          {\externalfigure[back-2.pdf][page=10,width=\distributedhsize\textwidth\emwidth4]}{Page 10}
          {\externalfigure[back-2.pdf][page=11,width=\distributedhsize\textwidth\emwidth4]}{Page 11}
          {\externalfigure[back-2.pdf][page=12,width=\distributedhsize\textwidth\emwidth4]}{Page 12}
    \stopcombination
\stopplacefigure

In the case of side floats the following cases occur. Of course multiple such
cases can follow each order so in practice we have to deal with an accumulation.

\startlinecorrection[blank]
\startMPcode
    linejoin := linecap := butt ;

    numeric u  ; u  := 1mm ;
    numeric lw ; lw := u/2 ;

    pickup pencircle scaled 2lw ;

    def example (expr n) (text t) (text l) =
        path b ; b := boundingbox image (
            for i=t : draw ( 0u,i*2u) -- (20u,i*2u) ; endfor ;
            for i=l : draw ( 0u,i*2u) -- (20u,i*2u) ; endfor ;
        ) ;
        picture p ; p := image (
            for i=t : draw ( 0u,i*2u) -- (20u,i*2u) ; endfor ;
            for i=l : draw (11u,i*2u) -- (20u,i*2u) ; endfor ;
        ) ;
        setbounds p to b ;
        path q ; q := unitsquare xysized(10u,10u) shifted (0,4u) ;
        draw image (
            fill boundingbox p leftenlarged -lw rightenlarged -lw withcolor "blue" ;
            draw p withcolor .5white ;
            fill q withcolor "red"  ;
            draw textext("\bf " & decimal n) shifted (center q) withcolor white ;
        ) shifted ((n-1)*30u,0) ;
    enddef ;

    example (1) (1)   (2,3,4)       ;
    example (2) (1,8) (2,3,4,5,6,7) ;
    example (3) (8)   (5,6,7)       ;
    example (4) ()    (3,4,5,6)     ;

    currentpicture := currentpicture ysized(3*LineHeight- StrutDepth) ;

\stopMPcode
\stoplinecorrection

As often in \TEX\ coming up with a solution is not a the problem but interference
is. You can cook up a solution for one case that fails in another. Backgrounds
fall into this category, as do side floats. In the next pages we will demonstrate
a few cases. In practice you can probably always come up with something that
works out well, but in an automated workflow (like unattended \XML\ to \PDF\
conversion) you can best play safe. We show some examples on the next pages.

\blank

\definetextbackground
  [demobg]
  [backgroundcolor=blue,
   color=white,
   frame=off,
   location=paragraph]

\setupfloatcaption
  [color=black]

\definesimulatewords
  [demo]
  [n=50,
   m=\simulatewordsparameter{n},
   min=1,
   max=5,
   color=text,
   line=yes,
   random=100]

\startbuffer
\placefigure
  [left]
  {case 1}
  {\blackrule[width=12cm,height=1cm,color=red]}
\simulatewords[demo][n=10]
\starttextbackground[demobg]
    \simulatewords[demo][n=30]
\stoptextbackground
\flushsidefloats

\blank

\starttextbackground[demobg]
    \simulatewords[demo][n=40]
    \placefigure
      [left]
      {case 2}
      {\blackrule[width=12cm,height=1cm,color=red]}
    \simulatewords[demo][n=40]
\stoptextbackground
\flushsidefloats

\blank

\placefigure
  [left]
  {case 3}
  {\blackrule[width=4cm,height=15mm,color=red]}
\starttextbackground[demobg]
    \simulatewords[demo][n=40]
\stoptextbackground
\simulatewords[demo][n=40]
\flushsidefloats

\blank

\simulatewords[demo][n=35]
\placefigure
  [left]
  {case 4}
  {\blackrule[width=4cm,height=1cm,color=red]}
\simulatewords[demo][n=20]
\starttextbackground[demobg]
    \simulatewords[demo][n=25]
\stoptextbackground
\simulatewords[demo][n=40]
\flushsidefloats

\blank

\stopbuffer

\start \setupwhitespace[none] \getbuffer \stop \blank

The previous examples were typeset with:

\typebuffer

Regular (page flow) floats are a different story. Here we have the problem that a
float might be postpones because there is no room on the current page and they
are moved forward (which is why they're called float). Again we show some
examples.

% \page

\startbuffer[sample]
One problem introduced by the internet is that one can view music online. Well,
it's actually not really a problem as it is fun to do, but it does interfere with
development of code: one can enter distraction mode quite easily.
\stopbuffer

\startbuffer
\starttextbackground[demobg]
    \par \getbuffer[sample] \par
    \placefigure{}{\blackrule[width=4cm,height=1cm,color=red]}
    \par \getbuffer[sample] \par
    \placefigure{}{\blackrule[width=4cm,height=3cm,color=red]}
    \par \getbuffer[sample] \par
    \placefigure{}{\blackrule[width=4cm,height=2cm,color=red]}
    \par \getbuffer[sample] \par
\stoptextbackground
\stopbuffer

\blank \getbuffer \blank

The input is:

\typebuffer

A combination of both background avoiding mechanisms is shown on the next page
(we flush a few more grapohics so that we cross a page boundary):

% \page

\startbuffer
\starttextbackground[demobg]
    \placefigure{}{\blackrule[width=4cm,height=2cm,color=red]}
    \par \input ward \par
    \placefigure[left]{}{\blackrule[width=4cm,height=2cm,color=red]}
    \par \input ward \par
    \placefigure{}{\blackrule[width=4cm,height=2cm,color=red]}
    \placefigure{}{\blackrule[width=4cm,height=2cm,color=red]}
    \placefigure{}{\blackrule[width=4cm,height=2cm,color=red]}
    \placefigure{}{\blackrule[width=4cm,height=2cm,color=red]}
    \placefigure{}{\blackrule[width=4cm,height=2cm,color=red]}
    \placefigure{}{\blackrule[width=4cm,height=2cm,color=red]}
    \par \input ward \par
    \placefigure{}{\blackrule[width=4cm,height=2cm,color=red]}
    \par \input ward \par
\stoptextbackground
\stopbuffer

\blank \getbuffer \blank

This is the result from:

\typebuffer

You can control the interaction between backgrounds and floars with the \type
{freeregion} parameter.

\startbuffer
\starttextbackground[demobg]
    \simulatewords[demo][n=40]
    \startplacefigure
      [location=left,
       title={free}]
        \blackrule[width=12cm,height=1cm,color=red]
    \stopplacefigure
    \simulatewords[demo][n=40]
    \startplacefigure
      [location=left,
       title={non|-|free},
       freeregion=no,
       color=textcolor]
        \blackrule[width=12cm,height=1cm,color=red]
    \stopplacefigure
    \simulatewords[demo][n=40]
    \startplacefigure
      [location=here,
       title={free}]
        \blackrule[width=12cm,height=1cm,color=red]
    \stopplacefigure
    \simulatewords[demo][n=40]
    \startplacefigure
      [location=here,
       title={non|-|free},
       freeregion=no,
       color=textcolor]
        \blackrule[width=12cm,height=1cm,color=red]
    \stopplacefigure
    \simulatewords[demo][n=40]
\stoptextbackground
\stopbuffer

\typebuffer

The next pages show the result, first with some tracing enabled sop that you
can see what gets freed. This visual effect is enabled with:

\starttyping
\enabletrackers[floats.freeregion]
\stoptyping

We now move to the next page.

\page
    \getbuffer
\page
    \enabletrackers[floats.freeregion]
    \getbuffer
    \disabletrackers[floats.freeregion]
\page

We have some control over side float placement and of course that will interfere
with backgrounds. Say that we have this:

\startbuffer
\definefloat
  [demofigureleft]
  [figure]
  [default=left,
  margin=1cm,
   leftmargindistance=2cm,
   rightmargindistance=2cm]

\definefloat
  [demofigureright]
  [demofigureleft]
  [default=right]
\stopbuffer

\typebuffer \getbuffer

Combined with the following we get the result on the next pages.

\startbuffer
\starttextbackground[demobg]
    \startplacefloat[figure][location=left]
        \blackrule[width=12cm,height=1cm,color=red]
    \stopplacefigure
    \simulatewords[demo][n=40]
    \blank
    \startplacefloat[figure][location=right]
        \blackrule[width=12cm,height=1cm,color=red]
    \stopplacefigure
    \simulatewords[demo][n=40]
    \blank
    \startplacefloat[demofigureleft]
        \blackrule[width=10cm,height=1cm,color=red]
    \stopplacefigure
    \simulatewords[demo][n=40]
    \blank
    \startplacefloat[demofigureright]
        \blackrule[width=10cm,height=1cm,color=red]
    \stopplacefigure
    \simulatewords[demo][n=40]
    \startplacefloat[figure] % [freeregion=no]
        \blackrule[width=12cm,height=1cm,color=red]
    \stopplacefigure
    \simulatewords[demo][n=40]
\stoptextbackground
\stopbuffer

\typebuffer

\page

\start
    \enabletrackers[floats.freeregion]
    \setupwhitespace[none]
    \getbuffer
    \disabletrackers[floats.freeregion]
\stop

\page

\start
    \setupwhitespace[none]
    \getbuffer
\stop

\page

\stop \stopchapter

\stopcomponent
